% !TEX TS-program = pdflatex
% !TEX encoding = UTF-8 Unicode

% This is a simple template for a LaTeX document using the "article" class.
% See "book", "report", "letter" for other types of document.

%%%%%%%%%%%%%%%%%%%%%%%%%%%%%%%%%%%%%
%
% Template para producir documentos "camera ready" para Anales de la AFA
% 
%  No se debería tener que modificar nada del siguiente preámbulo, donde están 
%  las customizaciones para adaptar la clase "article" a los Anales AFA.
%
%%%%%%%%%%%%%%%%%%%%%%%%%%%%%%%%%%%%%
\documentclass[10pt,twocolumn]{article} 

\usepackage[utf8]{inputenc} % set input encoding (not needed with XeLaTeX)
\usepackage[spanish]{babel}


%%% DIMENSIONES DE LA PÁGINA
\usepackage{geometry} 
\geometry{a4paper} 
 \geometry{top=2.25cm} 
 \geometry{bottom=2.25cm} 
 \geometry{left=2.5cm} 
 \geometry{right=2cm} 


%%% PACKAGES
\usepackage{graphicx} 
\usepackage{paralist} % very flexible & customisable lists (eg. enumerate/itemize, etc.)
\usepackage{verbatim} % adds environment for commenting out blocks of text & for better verbatim
\usepackage{subfig} % make it possible to include more than one captioned figure/table in a single float
\usepackage{lipsum}  
\usepackage{hyperref}
\usepackage[superscript]{cite}  %REFERENCIAS EN SUPERÍNDICE

% Ajusta los captions de tablas y figuras a italics
\usepackage[format=plain,
            labelfont=it,
            textfont=it]{caption}
% These packages are all incorporated in the memoir class to one degree or another...

%%% HEADERS & FOOTERS
\usepackage{fancyhdr} % This should be set AFTER setting up the page geometry
\pagestyle{fancy} % options: empty , plain , fancy
\renewcommand{\headrulewidth}{0pt} % customise the layout...
\lhead{}\chead{}\rhead{}
\lfoot{}\cfoot{\thepage}\rfoot{}
%\renewcommand{\thefootnote}{\fnsymbol{footnote}}

\usepackage{varwidth}
\usepackage{authblk}
\newcommand{\filiacion}[2]{\affil[#1]{\protect\begin{varwidth}[t]{\linewidth}\protect\centering \normalfont#2 \protect\end{varwidth}}}
\newcommand{\autor}[2]{\author[#1]{\bf #2}}
\newcommand{\corresponding}[2]{\author[#1]{\bf #2\thanks{}}}
\newcommand\cauthemail[1]{\footnotetext{#1}}
\newcommand{\fecha}[1]{\date{\vspace{-1ex}\small{#1}}}
\newcommand{\titulo}[2]{\title{\bf{\large{#1 \\ \vspace{1.5ex} #2 }}}}
\newcommand{\esresumen}[1]{\small{#1 \par}\vspace{1.5ex}}
\newcommand{\pclaves}[1]{\small{\emph{#1} \par}\vspace{1.5ex}}
\newcommand{\enresumen}[1]{\small{#1 \par}\vspace{1.5ex}}
\newcommand{\keywords}[1]{\small{\emph{#1} \par}\vspace{1.5ex}}



%%% APARIENCIA DE TITULOS, SECCIONES Y SUBSECCIONES
\usepackage{sectsty}
\allsectionsfont{\fontsize{10}{12}\sffamily\bfseries\upshape} % (See the fntguide.pdf for font help)

\usepackage{titlesec}
\titlespacing*{\section}{0pt}{1.5ex}{0.8ex}
\titlespacing*{\subsection}{0pt}{1.2ex}{0.6ex}
\setcounter{secnumdepth}{1}   %no numera las subsecciones

\usepackage[nottoc,notlof,notlot]{tocbibind} % Put the bibliography in the ToC
\usepackage[titles,subfigure]{tocloft} % Alter the style of the Table of Contents
\renewcommand{\cftsecfont}{\rmfamily\mdseries\upshape}
\renewcommand{\cftsecpagefont}{\rmfamily\mdseries\upshape} % No bold!
\renewcommand\thesection{\Roman{section}}
\renewcommand\thesubsection{}



%%% PARA EL FORMATO DE TABLAS
\usepackage{booktabs} % for much better looking tables
\usepackage{array} % for better arrays (eg matrices) in maths
\makeatletter
\newcommand{\thickhline}{%
    \noalign {\ifnum 0=`}\fi \hrule height 1.5pt
    \futurelet \reserved@a \@xhline
}
\newcolumntype{"}{@{\hskip\tabcolsep\vrule width 1pt\hskip\tabcolsep}}
\makeatother
\newcolumntype{L}[1]{>{\raggedright\let\newline\\\arraybackslash\hspace{0pt}}m{#1}}
\newcolumntype{C}[1]{>{\centering\let\newline\\\arraybackslash\hspace{0pt}}m{#1}}
\newcolumntype{R}[1]{>{\raggedleft\let\newline\\\arraybackslash\hspace{0pt}}m{#1}}

\renewcommand\spanishtablename{Tabla}  

% Corresponding author
\makeatletter
\renewcommand\@biblabel[1]{#1.}
\makeatother

\pagestyle{empty}

%%%%%% En principio no debería hacer falta modificar nada por encima de esta línea

%%%%%%%%%%%%%%%%%%%%%%%%%%%%%%%%%%%%%%%

%%% El contenido del documento comienza a partir de acá 

%título del trabajo: en el primer campo en castellano, en el segundo en inglés
\titulo{COMPARACIÓN DE TRES MÉTODOS DE DERIVACIÓN DE LA IRRADIANCIA SOLAR EFECTIVA PARA LA PRODUCCIÓN DE PRE-VITAMINA D\textsubscript{3} EN LA PIEL, EN LA CIUDAD DE ROSARIO, ARGENTINA}{COMPARISON OF THREE DERIVATION METHODS OF EFFECTIVE SOLAR IRRADIANCE FOR THE PRODUCTION OF PRE-VITAMIN D\textsubscript{3} ON THE SKIN, IN ROSARIO, ARGENTINA}

\autor{1}{M. Dávalos}
\autor{2}{A. Ipiña}
\autor{3}{G. López-Padilla}
\autor{2}{R. D. Piacentini}

%afiliaciones: se pueden combinar
\filiacion{1}{Investigadora Independiente, (64810) México}
\filiacion{2}{Instituto de Física Rosario (IFIR) – Universidad Nacional Rosario – Consejo Nacional de Investigaciones Científicas y
Técnicas, 27 de Febrero 210BIS – (S2000EKF) Rosario – Argentina.}
\filiacion{3}{Facultad de Ciencias Físico Matemáticas – Universidad Autónoma de Nuevo León, Pedro de Alba S/N - Ciudad
Universitaria San Nicolás de los Garza (66451) – Méxicoa.}

\fecha{Recibido: xx/xx/xx; Aceptado: xx/xx/xx} %No modificar


\setcounter{Maxaffil}{0}
\renewcommand\Affilfont{\itshape\small}


\begin{document}

\renewcommand{\abstractname}{}
\twocolumn[
  \begin{@twocolumnfalse}
   \maketitle
    \begin{abstract}\vspace{-12ex}
\centering\begin{minipage}{\dimexpr\paperwidth-6cm}

%Abstract en castellano
\esresumen{En los últimos años, el interés por el estudio de la vitamina D ha aumentado debido a la frecuencia incidente en
las personas que presentan deficiencia de esta vitamina, ya que muy pocos alimentos la contienen de manera
natural. Sin embargo, es posible generar la pre-vitamina D\textsubscript{3} a través de la piel cuando es expuesta a la radiación
solar ultravioleta. En este estudio se determina la irradiancia solar efectiva para la producción de pre-vitamina D\textsubscript{3}
en la ciudad de Rosario, Argentina, utilizando tres métodos: a) modelo TUV, b) fórmula de CIE-2014 sobre
mediciones de índice UV y c) ecuación de Herman. Se desarrolló un código en python para optimizar la descarga
de datos satelitales, calcular las integrales para obtener las dosis de la irradiancia pre-vitamina D\textsubscript{3} y eritémica, así
como los tiempos de exposición solar (TES). Además, compara los valores de dichas irradiancias en condiciones
de cielo despejado. Se discute la variación de los tiempos de exposición solar que alcanzan la dosis mínima de
pre-vitamina D\textsubscript{3} con una exposición del 25\% del cuerpo (cara, cuello y brazos).}
\pclaves{Palabras Clave: radiación solar UV, vitamina D, dosis, métodos, Argentina.} %palabras clave

%Abstract en inglés
\enresumen{In the last few years, the interest in the study of vitamin D has grown due to the frequent people showing
deficiency of this vitamin since very few foods contain it naturally. However, it is possible to generate
pre-vitamin D\textsubscript{3} through the skin when it is exposed to ultraviolet solar radiation. In this study we determine the
effective solar irradiance production of pre-vitamin D\textsubscript{3} in Rosario, Argentina using three methods: a) TUV model,
b) CIE-2014 formula applied on UV index measurements and c) Herman’s (2010) equation. We developed a
python code in order to optimize the download of satellite data, integrate the doses of pre-vitamin D\textsubscript{3} and
erythemic irradiance, as well as the solar exposure times (TES). In addition, the python script compares the values
of these irradiances in clear-sky conditions. We discuss the variation of the TES that reach the minimum dose of
pre-vitamin D\textsubscript{3} with a 25\% exposure of the body (face, neck and arms).}
\keywords{UV solar radiation, vitamin D, doses, methods, Argentina.}  %key words

\end{minipage}
\vspace{4ex}
 \end{abstract}
  \end{@twocolumnfalse}
]

\thispagestyle{empty}
\setcounter{footnote}{1}
\cauthemail{zgamma@citedef.gob.ar}  % Dirección de correo electrónico del corresponding author
aldds
\bibliographystyle{abbrv}
\renewcommand{\bibname}{}
 \nocite{*}
 \bibliography{referencias_anales}
\end{document}
